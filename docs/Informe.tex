\subsubsection{Modificaciones del Algoritmo}\subsubsection{Modificaciones del Algoritmo}\subsubsection{Modificaciones del Algoritmo}\subsubsection{Modificaciones del Algoritmo}\subsubsection{Modificaciones del Algoritmo}\subsubsection{Modificaciones del Algoritmo}\subsubsection{Modificaciones del Algoritmo}\subsubsection{Modificaciones del Algoritmo}\documentclass[a4paper, 11pt]{article}
\usepackage{amsmath}
\usepackage{amsfonts}
\usepackage{amssymb}
\usepackage{caratula}
\usepackage[spanish, activeacute]{babel}
\usepackage[usenames,dvipsnames]{color}
\usepackage[width=15.5cm, left=3cm, top=2.5cm, height= 24.5cm]{geometry}
\usepackage{graphicx}
\usepackage[utf8]{inputenc}
\usepackage{listings}
\usepackage[all]{xy}
\usepackage{multicol}
\usepackage{subfig}
\usepackage{algorithm}
\usepackage{algorithmic}
\usepackage{cancel}
\usepackage{float}
\usepackage{xcolor}
\usepackage{color,hyperref}
\setcounter{secnumdepth}{3} %%agrego subsubsection


%%%%%%%%%%%%%% ALGUNAS MACROS %%%%%%%%%%%%%%
% For \url{SOME_URL}, links SOME_URL to the url SOME_URL
\providecommand*\url[1]{\href{#1}{#1}}

\setlength{\parskip}{10pt plus 1pt minus 1pt}
\usepackage{tikz}
\def\checkmark{\tikz\fill[scale=0.4](0,.35) -- (.25,0) -- (1,.7) -- (.25,.15) -- cycle;}

% Same as above, but pretty-prints SOME_URL in teletype fixed-width font
\renewcommand*\url[1]{\href{#1}{\texttt{#1}}}

% Comando para poner el simbolo de Reales
\newcommand{\real}{\hbox{\bf R}}

\providecommand*\code[1]{\texttt{#1}}

%uso: \ponerGrafico{file}{caption}{scale}{label}
\newcommand{\ponerGrafico}[4]
{\begin{figure}[H]
	\centering
	\subfloat{\includegraphics[scale=#3]{#1}}
	\caption{#2} \label{fig:#4}
\end{figure}
}

\renewcommand{\algorithmiccomment}[1]{\hfill #1}

%%%%%%%%%%%%%%%%%%%%%%%%%%%%%%%%%%%%%%%%%%%%

\materia{Algoritmos y Estructuras de Datos III}

\titulo{TP2}
%\fecha{fecha de entrega}
%\grupo{Nro grupo}
\integrante{Ezequiel Aguerre}{246/07}{ezeaguerre@gmail.com}
\integrante{Juan Vanecek}{169//10}{juann.vanecek@gmail.com}
\integrante{Santiago Camacho}{110/09}{santicamacho90@gmail.com}
\integrante{Tomas Rodriguez}{527/10}{tomirodriguez.89@gmail.com}

\include{templates}

\begin{document}
\pagestyle{myheadings}
\maketitle
%\markboth{Nombre materia}{Nombre TP}

\thispagestyle{empty}
\tableofcontents

%\setcounter{section}{-1}

\newpage
\section{Problema 1: Roban\'umeros}

\subsection{Presentaci\'on del problema}

\subsection{Resoluci\'on}

\subsubsection{1era solucion}
En cada mano, agarrar la cantidad de cartas que sumen mas. 

No funciona. Contraejemplo: $[-1, -6, -10]$
  
\subsubsection{2da solucion}

Sea $v$ un vector de $n$ enteros que son los valores de las cartas, y dada la funcion recursiva: 

\begin{equation*}\hspace*{-2cm}
 f(i,j,p) = \max \left( 
  \underbrace{ \max_{1\leq k \leq j}{ \left\{ \left( \sum_{t=i}^k v_t \right) p + f\left(k,j, (p+1)\bmod{2} \right) \right\} } }_{ \text{Lo mejor de la izquierda} },
  \underbrace{ \max_{1\leq k \leq j}{ \left\{ \left( \sum_{t=k}^j v_t \right) p + f\left(i,k, (p+1)\bmod{2} \right) \right\} } }_{ \text{Lo mejor de la derecha} }
 \right) 
\end{equation*}

La soluci\'on al problema es $f(1,n,0)$ donde el 3er par\'ametro puede ser 0 o 1, y representa al jugador

% \left{ \right} 

\subsubsection{3ra solucion}

La solucion es el valor $\max (f_{\rightarrow}(1,n,yo), f_{\leftarrow}(1,n,yo))$ donde las funciones internas se definen como: 

\begin{align*}\hspace*{-2cm}
 f_{\rightarrow}(i,j,p) &= 
 \left\{
    \begin{array}{ll}
	\displaystyle \max_{1\leq k\leq j} \left\{ \sum_{t=i}^k v[t] + \min( f_{\rightarrow}(k+1, j, p.next()), f_{\leftarrow}(k+1,j,p.next() ))  \right\}& \mbox{si } p = yo \\
	\displaystyle \min_{1\leq k\leq j} \left\{ \max( f_{\rightarrow}(k+1, j, p.next()), f_{\leftarrow}(k+1,j,p.next() ) \right\} & \mbox{si } p = tu
    \end{array}
\right. \\
 f_{\leftarrow}(i,j,p) &= 
 \left\{
    \begin{array}{ll}
	\displaystyle \max_{1\leq k\leq j} \left\{ \sum_{t=k}^j v[t] + \min( f_{\rightarrow}(i, k-1, p.next()), f_{\leftarrow}(i, k-1,p.next() ))  \right\}& \mbox{si } p = yo \\
	\displaystyle \min_{1\leq k\leq j} \left\{ \max( f_{\rightarrow}(i, k-1, p.next()), f_{\leftarrow}(i, k-1,p.next() ) \right\} & \mbox{si } p = tu
    \end{array}
\right. \\
\end{align*}

\newpage
\subsubsection{4ta solucion}

Defino la funcion $f(i,j)$ como la solucion optima usando de las cartas $i$ a $j$. 

Esto es lo maximo que puedo agarrar con las cartas de la izquierda o de la derecha. 

Supongamos que por la izquierda lo mejor que puedo hacer es usando las primeras $k$ cartas. Significa que el valor que puedo tomar es $\sum_{t=i}^{k} carta[t]$ m\'as lo que me deja tomar el otro jugador (que va a jugar optimamente) en la mitad $[k+1, j]$. Esto es el total que suma las cartas en la mitad ya dicha, menos $f(k+1, j)$, ya que es el valor optimo. La funci\'on queda asi: 

\begin{eqnarray*}\hspace*{-4cm}
  f(i,j) &=& \max\left( 
      \max_{i \leq k \leq j}\left\{ \sum_{t=i}^k v[t] + \left( \sum_{t=k+1}^{j} v[t] - f(k+1,j) \right) \right\},  
      \max_{i \leq k \leq j}\left\{ \sum_{t=k}^j v[t] + \left( \sum_{t=i}^{k-1} v[t] - f(i,k-1) \right) \right\} 
    \right) \\
    &=& \max \left(
      \max_{i \leq k \leq j}\left\{ \sum_{t=i}^j v[t] - f(k+1,j) \right\},  
      \max_{i \leq k \leq j}\left\{ \sum_{t=i}^j v[t] - f(i,k-1) \right\} 
    \right) \\
    &=& \sum_{t=i}^j v[t] + \max \left(
      \max_{i \leq k \leq j}\left\{ - f(k+1,j) \right\},  
      \max_{i \leq k \leq j}\left\{ - f(i,k-1) \right\} 
    \right) \\
    &=& \sum_{t=i}^j v[t] - \min \left(
      \min_{i \leq k \leq j}\left\{ f(k+1,j) \right\},  
      \min_{i \leq k \leq j}\left\{ f(i,k-1) \right\} 
    \right) \\
    &=& \sum_{t=i}^j v[t] - \min_{i \leq k \leq j}\left\{ \min \left( f(k+1,j), f(i,k-1) \right) \right\} 
\end{eqnarray*}


\subsection{Demostraci\'on}

\subsection{An\'alisis de complejidad}

\subsection{Test de complejidad}

\subsection{Compilar y ejecutar}


\newpage
\section{Problema 2: La centralita (de gas)}

\subsection{Presentaci\'on del problema}

\subsection{Resoluci\'on}

\subsection{Demostraci\'on}

\subsection{An\'alisis de complejidad}

\subsection{Test de complejidad}

\subsection{Compilar y ejecutar}


\newpage
\section{Problema 3: Saltos en {\it La Matrix}}

\subsection{Presentaci\'on del problema}

La Matrix es un juego que consite de participantes en un tablero cuadrado. Cada casillero contiene un resorte que le permite al jugador saltar a otro casillero en direcci\'on vertical u horizontal dependiendo del valor de salto del resorte.

El problema esta en buscar la cantidad m\'inima de saltos que puede hacer un participante desde un casillero origen a otro destino. Ademas cada participante tiene una cantidad de potencias extra que puede usar para llegar a casilleros mas alejados.

\subsection{Resoluci\'on}

Para resolver este problema se plante\'o hacer un grafo \textit{tridimensional} (explicado m\'as adelante). 

Al plantear el grafo (dirigido) de casilleros como nodos conectados a sus posibles casilleros de saltos hac\'ia f\'acil la busqueda de camino m\'inimo entre origen y destino, pero se comlpicaba el c\'alculo del uso de las potencias, ya que era muy costoso saber cuantas potencias se habian usado.

Es por eso que se plante\'o finalmente uno tridimensional. Este incluye informaci\'on de las potencias en s\'i mismo.
El grafo dirigido tridimensional consiste en repetir el grafo de casilleros como nodos y conexiones como aristas $k$ cantidad de veces, donde $k =$ n\'umero de potencias.
Es decir, cuando se toma una decisi\'on que no requiere potencias, me quedo en la misma matriz, pero cuando uso alguna potencia, salto a otra matriz dependiendo de la cantidad de potencias utilizadas.
No puedo saltar de una matriz con mayor $k$ a una de menor.
Entonces obtengo un grafo dirigido de $(n^2)*k$ nodos, donde una misma posici\'on esta repetida $k$ veces.

\subsubsection{Algoritmo}

Como cada arista representa un salto, todas tienen el mismo costo. Por eso, adaptamos el problema a buscar el camino m\'inimo con el algoritmo de BFS en un grafo dirigido, empezando desde el casillero origen.

\begin{verbatim}
LaMatrix(Tablero inicio fin potencia)
    niveles = { inicio : 0 } //diccionario con la clave inicio y 0 como significado
    anterior = { inicio : NULL } //diccionario, clave inicio y NULL como significado
    i = 1
    frontera = [inicio]
    while(frontera){
        siguiente = []
        for x in frontera {
            for y in vecinos(x){ //vecinos es un arreglo del alcance que tiene x
                if !nivel.estaDefinido(y){
                    nivel[y] = i
                    anterior[y] = x
                    siguiente.agregar(y)
                }
            }
        }
        frontera = siguitente
        i++
        }

    saltosAfin = niveles.significado(fin) //cantidad de saltos
    secuenciaDeSaltos = anterior.dameSecuencia(fin) //devuelve el camino desde el inicio hasta en fin.
    return (saltosAfin, secuenciaDeSaltos)
\end{verbatim}


\subsection{Demostraci\'on} 

\subsection{An\'alisis de complejidad}

El algoritmo es principalmente una busqueda en anchura, y esta tiene complejidad temporal de $O(V + E)$ donde $V$ son los nodos y $E$ las aristas.
En nuestro caso tenemos $(n^2)*k$ nodos y en caso m\'aximo en que todos los nodos esten conectados a sus posibles lugares de salto serian $(n^2)*2(n-1)*k$ ya que hay $n^2$ nodos con $n-1$ conexiones m\'aximas en cada direcci\'on (vertical u horizontal) por cada nivel $k$. Esto dar\'ia acotado superiormente $(n^3)*k$.
Es decir que el peor caso ser\'ia de $O((n^2)*k + (n^3)*k)$ que es lo mismo que $O((n^3)*k)$.

%\subsection{Test de complejidad}

%\subsection{Compilar y ejecutar}

















\end{document}
