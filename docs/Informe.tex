\documentclass[a4paper, 11pt]{article}
\usepackage{amsmath}
\usepackage{amsfonts}
\usepackage{amssymb}
\usepackage{caratula}
\usepackage[spanish, activeacute]{babel}
\usepackage[usenames,dvipsnames]{color}
\usepackage[width=15.5cm, left=3cm, top=2.5cm, height= 24.5cm]{geometry}
\usepackage{graphicx}
\usepackage[utf8]{inputenc}
\usepackage{listings}
\usepackage[all]{xy}
\usepackage{multicol}
\usepackage{subfig}
\usepackage{algorithm}
\usepackage{algorithmic}
\usepackage{cancel}
\usepackage{float}
\usepackage{xcolor}
\usepackage{color,hyperref}
\setcounter{secnumdepth}{3} %%agrego subsubsection


%%%%%%%%%%%%%% ALGUNAS MACROS %%%%%%%%%%%%%%
% For \url{SOME_URL}, links SOME_URL to the url SOME_URL
\providecommand*\url[1]{\href{#1}{#1}}

\setlength{\parskip}{10pt plus 1pt minus 1pt}
\usepackage{tikz}
\def\checkmark{\tikz\fill[scale=0.4](0,.35) -- (.25,0) -- (1,.7) -- (.25,.15) -- cycle;}

% Same as above, but pretty-prints SOME_URL in teletype fixed-width font
\renewcommand*\url[1]{\href{#1}{\texttt{#1}}}

% Comando para poner el simbolo de Reales
\newcommand{\real}{\hbox{\bf R}}

\providecommand*\code[1]{\texttt{#1}}

%uso: \ponerGrafico{file}{caption}{scale}{label}
\newcommand{\ponerGrafico}[4]
{\begin{figure}[H]
	\centering
	\subfloat{\includegraphics[scale=#3]{#1}}
	\caption{#2} \label{fig:#4}
\end{figure}
}

\renewcommand{\algorithmiccomment}[1]{\hfill #1}

%%%%%%%%%%%%%%%%%%%%%%%%%%%%%%%%%%%%%%%%%%%%

\materia{Algoritmos y Estructuras de Datos III}

\titulo{TP2}
%\fecha{fecha de entrega}
%\grupo{Nro grupo}
\integrante{Ezequiel Aguerre}{246/07}{ezeaguerre@gmail.com}
\integrante{Juan Vanecek}{169//10}{juann.vanecek@gmail.com}
\integrante{Santiago Camacho}{110/09}{santicamacho90@gmail.com}
\integrante{Tomas Rodriguez}{527/10}{tomirodriguez.89@gmail.com}

\include{templates}

\begin{document}
\pagestyle{myheadings}
\maketitle
%\markboth{Nombre materia}{Nombre TP}

\thispagestyle{empty}
\tableofcontents

%\setcounter{section}{-1}

\newpage
\section{Problema 1: Roban\'umeros}

\subsection{Presentaci\'on del problema}

\subsection{Resoluci\'on}

\subsubsection{1era solucion}
En cada mano, agarrar la cantidad de cartas que sumen mas. 

No funciona. Contraejemplo: $[-1, -6, -10]$
  
\subsubsection{2da solucion}

Sea $v$ un vector de $n$ enteros que son los valores de las cartas, y dada la funcion recursiva: 

\begin{equation*}\hspace*{-2cm}
 f(i,j,p) = \max \left( 
  \underbrace{ \max_{1\leq k \leq j}{ \left\{ \left( \sum_{t=i}^k v_t \right) p + f\left(k,j, (p+1)\bmod{2} \right) \right\} } }_{ \text{Lo mejor de la izquierda} },
  \underbrace{ \max_{1\leq k \leq j}{ \left\{ \left( \sum_{t=k}^j v_t \right) p + f\left(i,k, (p+1)\bmod{2} \right) \right\} } }_{ \text{Lo mejor de la derecha} }
 \right) 
\end{equation*}

La soluci\'on al problema es $f(1,n,0)$ donde el 3er par\'ametro puede ser 0 o 1, y representa al jugador

% \left{ \right} 

\subsubsection{3ra solucion}

La solucion es el valor $\max (f_{\rightarrow}(1,n,yo), f_{\leftarrow}(1,n,yo))$ donde las funciones internas se definen como: 

\begin{align*}\hspace*{-2cm}
 f_{\rightarrow}(i,j,p) &= 
 \left\{
    \begin{array}{ll}
	\displaystyle \max_{1\leq k\leq j} \left\{ \sum_{t=i}^k v[t] + \min( f_{\rightarrow}(k+1, j, p.next()), f_{\leftarrow}(k+1,j,p.next() ))  \right\}& \mbox{si } p = yo \\
	\displaystyle \min_{1\leq k\leq j} \left\{ \max( f_{\rightarrow}(k+1, j, p.next()), f_{\leftarrow}(k+1,j,p.next() ) \right\} & \mbox{si } p = tu
    \end{array}
\right. \\
 f_{\leftarrow}(i,j,p) &= 
 \left\{
    \begin{array}{ll}
	\displaystyle \max_{1\leq k\leq j} \left\{ \sum_{t=k}^j v[t] + \min( f_{\rightarrow}(i, k-1, p.next()), f_{\leftarrow}(i, k-1,p.next() ))  \right\}& \mbox{si } p = yo \\
	\displaystyle \min_{1\leq k\leq j} \left\{ \max( f_{\rightarrow}(i, k-1, p.next()), f_{\leftarrow}(i, k-1,p.next() ) \right\} & \mbox{si } p = tu
    \end{array}
\right. \\
\end{align*}

\newpage
\subsubsection{4ta solucion}

Defino la funcion $f(i,j)$ como la solucion optima usando de las cartas $i$ a $j$. 

Esto es lo maximo que puedo agarrar con las cartas de la izquierda o de la derecha. 

Supongamos que por la izquierda lo mejor que puedo hacer es usando las primeras $k$ cartas. Significa que el valor que puedo tomar es $\sum_{t=i}^{k} carta[t]$ m\'as lo que me deja tomar el otro jugador (que va a jugar optimamente) en la mitad $[k+1, j]$. Esto es el total que suma las cartas en la mitad ya dicha, menos $f(k+1, j)$, ya que es el valor optimo. La funci\'on queda asi: 

\begin{eqnarray*}\hspace*{-4cm}
  f(i,j) &=& \max\left( 
      \max_{i \leq k \leq j}\left\{ \sum_{t=i}^k v[t] + \left( \sum_{t=k+1}^{j} v[t] - f(k+1,j) \right) \right\},  
      \max_{i \leq k \leq j}\left\{ \sum_{t=k}^j v[t] + \left( \sum_{t=i}^{k-1} v[t] - f(i,k-1) \right) \right\} 
    \right) \\
    &=& \max \left(
      \max_{i \leq k \leq j}\left\{ \sum_{t=i}^j v[t] - f(k+1,j) \right\},  
      \max_{i \leq k \leq j}\left\{ \sum_{t=i}^j v[t] - f(i,k-1) \right\} 
    \right) \\
    &=& \sum_{t=i}^j v[t] + \max \left(
      \max_{i \leq k \leq j}\left\{ - f(k+1,j) \right\},  
      \max_{i \leq k \leq j}\left\{ - f(i,k-1) \right\} 
    \right) \\
    &=& \sum_{t=i}^j v[t] - \min \left(
      \min_{i \leq k \leq j}\left\{ f(k+1,j) \right\},  
      \min_{i \leq k \leq j}\left\{ f(i,k-1) \right\} 
    \right) \\
    &=& \sum_{t=i}^j v[t] - \min_{i \leq k \leq j}\left\{ \min \left( f(k+1,j), f(i,k-1) \right) \right\} 
\end{eqnarray*}


\subsection{Demostraci\'on}

\subsection{An\'alisis de complejidad}

\subsection{Test de complejidad}

\subsection{Compilar y ejecutar}


\newpage
\section{Problema 2: La centralita (de gas)}

\subsection{Presentaci\'on del problema}
En una regi\'on del pa\'is se est\'a considerando realizar una inversi\'on fuerte para proveer de gas natural
a un conjunto de pueblos que no disponen a\'un de este recurso. Para ello, es posible ubicar centrales
distribuidoras de gas en algunos de los pueblos y construir tuber\'ias para distribuir el gas de un pueblo a
otro. Un pueblo ser\'a provisto de gas siempre que exista alg\'un camino por medio de tuber\'ia hasta alguna
de las centrales (incluso si este camino pasa por otros pueblos). Debido al elevado costo de construcci\'on
de las centrales distribuidoras, el presupuesto con el que se cuenta alcanza para construir a lo sumo k
centrales.

Por otro lado, los ingenieros a cargo de este proyecto saben que mientras m\'as larga sea una tuber\'ia
construida entre dos pueblos, mayor es el riesgo de roturas y escapes de gas durante el trayecto (la
longitud de una tuber\'ia que conecta dos pueblos est\'a dada por la distancia entre estos dos pueblos).
En este sentido, se defini\'o el riesgo asociado a cada posible plan de construcci\'on como la mayor de las
longitudes de las tuber\'ias construidas en dicho plan.

Se pide escribir un algoritmo que determine un plan de construcci\'on de tuber\'ias y centrales (a lo sumo k
centrales) de forma tal que ning\'un pueblo quede sin acceso al preciado recurso. El plan debe indicar en
qu\'e pueblos se instalar\'an centrales y entre qu\'e pares de pueblos se construir\'an tuber\'ias de distribuci\'on de
gas. El plan propuesto debe tener riesgo m\'inimo, y en caso de haber m\'as de un plan \'optimo, el algoritmo
puede devolver cualquiera de ellos. Se pide que el algoritmo desarrollado tenga una complejidad temporal
de peor caso de $O(n^2)$, donde $n$ es la cantidad de pueblos del problema.

\subsection{Resoluci\'on}

\subsection{Pseudoc\'odigo}
\begin{verbatim}
resolver()
   construir agm del grafo
   ordenar aristas del agm de mayor a menor
   por cada arista del agm de mayor a menor y mientras haya centrales para colocar:
      puebloA := el pueblo de un extremo de la arista
      puebloB := el pueblo del otro extremo de la arista
      si hay mas de una central para colocar:
         colocar central en puebloA si no tiene
         colocar central en puebloB si no tiene
         eliminar arista del agm
      sino:
         si no hay centrales colocadas:
            colocar central en puebloA
         sino, si no hay central construida en puebloA ni en puebloB:
            salir
         sino, si sólo uno entre puebloA y puebloB tiene una central construida:
            si puebloA no tiene central:
               construir central en puebloA
            sino:
               construir central en puebloB
            eliminar arista del agm
         sino, los dos tienen central:
            eliminar arista del agm

\end{verbatim}

Construír un AGM a partir de una matriz de adyacencia utilizando Prim es $O(n^2)$ \footnote{http://en.wikipedia.org/wiki/Prim\%27s\_algorithm\#Time\_complexity}.
El AGM tiene $n-1$ aristas \footnote{http://en.wikipedia.org/wiki/Tree\_\%28graph\_theory\%29\#Definitions} (pues es un árbol), de modo que ordenar
todas las artistas tiene, en el peor caso, una complejidad de $O(n^2)$, incluso podría ser $O(n \log n)$, pero aunque así no sea no va a cambiar el resultado.
Por último, recorrer todas las aristas ordenadas es $O(n)$, y eliminar una arista del agm puede ser, según la implementación, entre $O(1)$ y $O(n)$, aún así, la complejidad de todo el ciclo no sería peor que $O(n^2)$. Sumando todo obtenemos una complejidad del algoritmo de $O(n^2)$.

\subsection{Demostraci\'on}

\subsection{An\'alisis de complejidad}

\subsection{Test de complejidad}

\subsection{Compilar y ejecutar}
Desde el directorio src/Ejercicio2:
\begin{itemize}
   \item {\bf Compilar:} ./ej1 make
   \item {\bf Ejecutar:} ./ej1
   \item {\bf Correr casos de test:} ./ej1 tests
   \item {\bf Benchmarks:} ./ej1 bench
\end{itemize}


\newpage
\section{Problema 3: Saltos en {\it La Matrix}}

\subsection{Presentaci\'on del problema}

La Matrix es un juego que consite de participantes en un tablero cuadrado de $n\times n$. Cada casillero contiene un resorte que le permite al jugador saltar a otro casillero en direcci\'on vertical u horizontal dependiendo del valor de salto del resorte. Adem\'as cada participante cuenta con una cantidad $k$ de potencias extra que puede usar para llegar a casilleros mas alejados, y que se van agotando a medida que las vamos usando.

El problema esta en encontrar la cantidad m\'inima de saltos que puede hacer un participante desde un casillero origen a otro destino, devolver la secuencia de casilleros por los que pasa, y hacerlo en un tiempo de $O(n^3 k)$. 

\subsection{Resoluci\'on}

El problema lo modelamos con un grafo, donde cada casillero es un nodo del mismo, y existe arista entre dos nodos, si y solo si son alcanzables en un salto. Ahora bien, para determinar si existe un salto entre dos casilleros hay que tener en cuenta las potencias extras que poseemos al momento de realizar el salto. A pesar de ello, las aristas no tienen pesos, y por eso usando un algoritmo como BFS podemos determinar la distancia m\'inima entre dos nodos y es este el que vamos a usar para encontrar la soluci\'on al problema propuesto. 

Asique lo que usamos es un BFS sobre un grafo \textit{din\'amico}, ya que el conjunto de nodos adyacentes a un nodo $v$ va a ir cambiando dependiendo de las potencias que tenga en ese momento, aunque como mucho puede en un casillero puedo saltar a todos los de arriba, a todos los de abajo y a todos los de los costados. Es decir, $d(v) \leq 2(n-1)$. 

Veamos que la complejidad pedida tambi\'en se cumple, ya que el grafo tiene $n^2$ nodos, y las aristas son: 

\begin{equation*}
 m = \sum_{v \in V} d(v) / 2 \leq \sum_{v \in V} 2(n-1) / 2 = n^2 (n-1) = n^3 - n^2
\end{equation*}

Por lo tanto, la complejidad del BFS queda en $O(|V| + |E|) \leq O(n^2 + n^3 - n^2) = O(n^3) \in O(n^3 k)$. 

\subsection{Implementaci\'on}

El algoritmo lo escribimos en Java, y contamos con dos clases: \code{Nodo} y \code{Tablero}. La primera cuenta con las siguientes variables: 

\begin{itemize}
 \item \code{int fila} y \code{col}: fila y columna del casillero.
 \item \code{int poderes\_disponibles}: variable que consulto cuando pido tomo sus adyacentes durante el BFS
 \item \code{int poderes\_usados}: variable que consulto al momento de construir la soluci\'on final. 
 \item \code{Nodo parent}: nodo padre en el \'arbol que arma BFS. Es null si el nodo en cuesti\'on es el root, o lo que es lo mismo, el nodo origen. 
\end{itemize}

Por otro lado, \code{Tablero} cuenta con las siguientes variables: 

\begin{itemize}
 \item \code{int n}: Cantidad de filas (y columnas)
 \item \code{int[][] potencias}: cantidad fija de potencias para cada casillero
 \item \code{boolean[][] recorridos}: a medida que BFS recorre el arbol de nodos, va seteando esta matriz sobre los nodos recorridos. 
 \item \code{Nodo src}: nodo origen, o casillero inicial. 
 \item \code{Nodo dst}: nodo destino, o casillero final. 
 \item \code{List<Nodo> solucion}: lista de nodos (o casilleros) en orden pertenecientes al camino m\'inimo entre \code{src} y \code{dst}
\end{itemize}

El BFS implementado funciona de la misma manera que \'el com\'un: arranco con el nodo inicial, recorro todos los adyacentes buscando el nodo final, si no lo encuentra, busca en todos los adyacentes que recorri\'o en la iteraci\'on anterior, y as\'i sucesivamente. Como vimos en la pr\'actica, BFS recorre todos los nodos del \'arbol y por lo tanto, si \code{dst} se encuentra en el mismo, nuestro algoritmo termina. 

Ahora queremos ver que no nos salteamos ninguna soluci\'on, y para ello veamos que siempre recorremos todas las posibilidades cuando buscamos los nodos adyacentes. Supongamos que un momento dado queremos encontrar los vecinos de un nodo $v_i$, la potencia del mismo que es $p_i$ y la potencia extra disponible (\code{poderes\_disponibles}) es $k_i$. Los casilleros a los que puedo llegar usando solamente $p_i$, van a seguir teniendo como \code{poderes\_disponibles} a $k_i$, en cambio, a los que llegu\'e usando $q$ poderes extra, van a tener como \code{poderes\_disponibles} a $k_i-q$. Adem\'as todos estos nodos adyacentes van a tener como \code{parent} a $v_i$. 

Nuestro BFS empieza con $v_1$ (\code{src}), cuya potencia fija es $p_1$, y su variable \code{poderes\_disponibles} seteado en $k$. Asique, al final, estamos cubriendo todos los posibles nodos adyacentes, y al final BFS recorre todos los posibles caminos que existen entre dos nodos cualesquiera. 

Una vez que encontramos a \code{dst} (sabemos que lo hace porque BFS termina), nuestro algoritmo ``sube'' en el \'arbol que construyo BFS a trav\'es de la variable de los nodos \code{parent} que apunta a su nodo padre, y va calculando la potencia extra usada en cada salto, ya que si un nodo $v_{i+1}$ tiene una \code{potencia\_disponible} de $k_{i+1}$ y su padre tiene una potencia de $k_i$, entonces la potencia extra (\code{poderes\_usados}) que us\'o el padre fue $k_{i+1} - k_i$.

\subsection{Algoritmo}

\begin{algorithm}[H]
\caption{\code{BFS(List<Nodo> nodos, int dist\_root)}}\label{max_pb}
\begin{algorithmic}[1]
\STATE \code{List<Nodo> nodos\_siguientes = []}
\FOR{\code{each nodo in nodos}}
  \IF{ \code{!fueRecorrido(nodo)} }
    \IF{ \code{nodo == dst} }
      \STATE \code{contruir\_camino(nodo)}
      \RETURN \code{dist\_root}
    \ELSE
      \STATE \code{marcarComoRecorrido(nodo)}
      \STATE \code{nodos\_siguientes.addAll( nodo.getAdyacentes() )}
    \ENDIF
  \ENDIF
\ENDFOR
\RETURN \code{BFS( nodos\_siguientes, dist\_root++ )}
\end{algorithmic}
\end{algorithm}

\subsubsection{Algoritmo}

%\subsection{Test de complejidad}

%\subsection{Compilar y ejecutar}


\end{document}
