\section{Problema 1: Roban\'umeros}

\subsection{Presentaci\'on del problema}

\subsection{Resoluci\'on}

Defino la funcion $f(i,j)$ como la solucion \'optima usando de las cartas $i$ a $j$. Esto es, lo m\'aximo que puedo agarrar con las cartas de la izquierda (1*) o de la derecha (2*). 

\begin{enumerate}
\item[(1*)] \label{x_izq} Supongamos que por la izquierda lo mejor que puedo hacer es usando las primeras $k$ cartas. Significa que el valor que puedo tomar es $\sum_{t=i}^{k} carta[t]$ m\'as lo que me deja tomar el otro jugador (que va a jugar de manera \'optima) en la mitad $[k+1, j]$. Esto es el total que suma las cartas en dicha mitad, menos $f(k+1, j)$, ya que es el valor \'optimo y el puntaje que va a juntar el otro jugador. 

\item[(2*)] \label{x_der} Asimismo, si lo mejor por derecha es usando $k$ cartas, entonces el puntaje \'optimo es la suma del valor de esas cartas ($\sum_{t=k}^{n} carta[t]$) m\'as lo que puedo tomar del lado $[i, k-1]$ despu\'es de que haya jugado el otro jugador. Y como este lo va a hacer \'optimamente, el valor que a m\'i me queda por elegir es $\sum_{t=i}^{k-1} carta[t] - f(i,k-1)$. 
\end{enumerate}

Es decir, que $f$ va a buscar el $k$ (entre $i$ y $j$) tal que maximiza el puntaje por izquierda (1*), y el de la derecha (2*), y se va a quedar con el m\'aximo de estos dos.

Matem\'aticamente, la funci\'on queda definida de la siguiente manera (para cuando $i \leq j$): 

$$\hspace*{-1cm} f(i,j) = \max\left( 
      \max_{i \leq k \leq j}\left\{ \sum_{t=i}^k v[t] + \left( \sum_{t=k+1}^{j} v[t] - f(k+1,j) \right) \right\},  
      \max_{i \leq k \leq j}\left\{ \sum_{t=k}^j v[t] + \left( \sum_{t=i}^{k-1} v[t] - f(i,k-1) \right) \right\} 
    \right) $$

Para $i > j$ $f(i,j) = 0$.
$v$ el vector con los valores de las cartas.
Si desarrollamos esta funci\'on nos queda la siguiente expresi\'on. 

\begin{eqnarray*}\hspace*{-4cm}
  f(i,j) &=& \max\left( 
      \max_{i \leq k \leq j}\left\{ \sum_{t=i}^k v[t] + \left( \sum_{t=k+1}^{j} v[t] - f(k+1,j) \right) \right\},  
      \max_{i \leq k \leq j}\left\{ \sum_{t=k}^j v[t] + \left( \sum_{t=i}^{k-1} v[t] - f(i,k-1) \right) \right\} 
    \right) \\
    &=& \max \left(
      \max_{i \leq k \leq j}\left\{ \sum_{t=i}^j v[t] - f(k+1,j) \right\},  
      \max_{i \leq k \leq j}\left\{ \sum_{t=i}^j v[t] - f(i,k-1) \right\} 
    \right) \\
    &=& \sum_{t=i}^j v[t] + \max \left(
      \max_{i \leq k \leq j}\left\{ - f(k+1,j) \right\},  
      \max_{i \leq k \leq j}\left\{ - f(i,k-1) \right\} 
    \right) \\
    &=& \sum_{t=i}^j v[t] - \min \left(
      \min_{i \leq k \leq j}\left\{ f(k+1,j) \right\},  
      \min_{i \leq k \leq j}\left\{ f(i,k-1) \right\} 
    \right) \\
    &=& \sum_{t=i}^j v[t] - \min_{i \leq k \leq j}\left\{ \min \left( f(k+1,j), f(i,k-1) \right) \right\} 
\end{eqnarray*}

%Veamos que el caso base es cuando hay dos cartas solamente, y $f(i,j)$ queda como 
Veamos un ejemplo con dos cartas

\begin{eqnarray*}
f(1,2) &=& v[1]+v[2] - \min( \min( f(2,2), f(1,0) ), \min( f(3,2), f(1,1) ) ) \\
       &=& v[1]+v[2] - \min( \min( f(2,2), 0 ), \min( 0, f(1,1) ) ) \\
       &=& v[1]+v[2] - \min\{ f(2,2), f(1,1), 0 \} = \\
		 &=& v[1]+v[2] - \min\{ v[2], v[1], 0 \}
\end{eqnarray*}

Si $v[1] \ge 0$ y $v[2] \ge 0$ entonces $v[1]+v[2] - \min\{v[2],v[1],0\} = v[1]+v[2]$ que es el máximo que se puede obtener.
Si $v[1] < 0$ entonces $v[1]+v[2] - \min\{v[2],v[1],0\} = v[1]+v[2]-v[1]=v[2]$,
que es positivo y es el óptimo. Lo mismo para $v[2] < 0$.
Si $v[1] < 0$ y $v[2] < 0$ entonces $v[1]+v[2] - \min\{v[2],v[1],0\} = v[1] ~ o ~ v[2]$, dependiendo cuál sea el máximo, que es, claramente la solución
óptima, ya que obtenemos el mejor puntaje posible (agarrar la carta más grande).

%Dado el caso base de las dos cartas, y la funci\'on recursiva $f(i,j)$, la funci\'on que devuelve la soluci\'on \'optima dadas las $n$ cartas es $f(1,n)$. 
\subsection{Pseudoc\'odigo}
Para resolverlo de manera iterativa y construyendo la solución desde ``abajo'' se recorre una matriz
en diagonal desde la diagonal principal, donde cada elemento (i,j) representa el puntaje obtenido
al tomar desde la carta i hasta la j de manera óptima. Al finalizar de construir la matríz el puntaje
óptimo se encuentra en la celda (0,n-1)
Para construir cada celda (i,j) necesitamos los elementos:

\begin{itemize}
   \item $(k+1,j) \forall i \le k < j$
   \item $(i,k-1) \forall i < k \le j$
\end{itemize}

Esto es así porque necesitamos saber qué puntaje obtendría el otro jugador si le dejamos esas cartas,
de manera de poder dejarle el menor puntaje posible.

\begin{verbatim}
calcular( cartas : array )
   n := cantidad de elementos en el arreglo
   cache := array(n*n)
   jugadas := array(n*n)
   para todo 0 <= i < n
      cache[i][i] := cartas[i]
      jugadas[i][i] := Jugada<izquierda,1 carta>

   para 1 <= columna < n
      para 0 <= diagonal < n - columna
         i := diagonal
         j := columna + diagonal
         min_ij := +Infinito
         puntos := 0
         para i <= k <= j
            a := cache[k+1][j] o 0 si k+1 > j
            b := cache[i][k-1] o 0 si k-1 < j
            si b < min_ij o a < min_ij
               si a < b
                  jugadas[i][j] = Jugada<izquierda,k-1+1 cartas>
                  min_ij = a
               sino
                  jugadas[i][j] = Jugada<derecha,j-k+1 cartas>
                  min_ij = b
            cache[i][j] = puntos - min_ij
   devolver optimo puntaje := cache[0][n-1]
   devolver lista de jugadas := reconstruir jugadas a partir de jugadas[0][n-1]
\end{verbatim}


\subsection{An\'alisis de complejidad}
Son simplemente 3 ciclos anidados que como mucho recorren $n$ elementos, el resto de las operaciones es $O(1)$, por
ende la complejidad es $O(n^3)$.

\subsection{Test de complejidad}

\subsection{Compilar y ejecutar}

Desde el directorio src/Ejercicio1:

\begin{itemize}
   \item {\bf Compilar:} ./ej1 make
   \item {\bf Ejecutar:} ./ej1
   \item {\bf Correr casos de test:} ./ej1 tests
   \item {\bf Benchmarks:} ./ej1 bench
\end{itemize}

\subsection{C\'odigo Fuente}
% Cambiar por el que resalta la sintaxis
\begin{verbatim}
public void calcularSolucion() {
 int n = cartas.length;
 for ( int i = 0; i < n; i++ ) {
  cache[i][i] = cartas[i];
  jugadas[i][i].desdeIzquierda( 1 );
 }

 for ( int columna = 1; columna < n; columna++ ) {
  for ( int diagonal = 0; diagonal < n - columna; diagonal++ ) {
   int i = diagonal;
   int j = columna + diagonal;
   int min_ij = Integer.MAX_VALUE;
   int puntos = 0;
   int x = i, y = j;
   for ( int k = i; k <= j; k++ ) {
    int a = 0, b = 0;
    puntos += cartas[k];
    if ( k + 1 <= j ) a = cache[k + 1][j];
    if ( k - 1 >= i ) b = cache[i][k - 1];
    if ( min_ij < a && min_ij < b ) continue;
    if ( a < b ) {
     jugadas[i][j].desdeIzquierda( k - i + 1 );
     min_ij = a;
    } else {
     jugadas[i][j].desdeDerecha( j - k + 1 );
     min_ij = b;
    }
   }
   cache[i][j] = puntos - min_ij;
  }
 }
}
\end{verbatim}
