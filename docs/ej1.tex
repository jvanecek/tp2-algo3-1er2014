\section{Problema 1: Roban\'umeros}

\subsection{Presentaci\'on del problema}

\subsection{Resoluci\'on}

Defino la funcion $f(i,j)$ como la solucion \'optima usando de las cartas $i$ a $j$. Esto es, lo m\'aximo que puedo agarrar con las cartas de la izquierda (1*) o de la derecha (2*). 

\begin{enumerate}
\item[(1*)] \label{x_izq} Supongamos que por la izquierda lo mejor que puedo hacer es usando las primeras $k$ cartas. Significa que el valor que puedo tomar es $\sum_{t=i}^{k} carta[t]$ m\'as lo que me deja tomar el otro jugador (que va a jugar \'optimamente) en la mitad $[k+1, j]$. Esto es el total que suma las cartas en dicha mitad, menos $f(k+1, j)$, ya que es el valor \'optimo y el puntaje que va a juntar el otro jugador. 

\item[(2*)] \label{x_der} As\'i mismo, si lo mejor por derecha es usando $k$ cartas, entonces el puntaje \'optimo es la suma del valor de esas cartas ($\sum_{t=k}^{n} carta[t]$) m\'as lo que puedo tomar del lado $[i, k-1]$ despu\'es de que haya jugado el otro jugador. Y como este lo va a hacer \'optimamente, el valor que a m\'i me queda por elegir es $\sum_{t=i}^{k-1} carta[t] - f(i,k-1)$. 
\end{enumerate}

Es decir, que $f$ va a buscar el $k$ (entre $i$ y $j$) tal que maximiza el puntaje por izquierda (1*), y el de la derecha (2*), y se va a quedar con el m\'aximo de estos dos.

Matem\'aticamente, la funci\'on queda definida de la siguiente manera (para cuando $i \leq j$): 

$$\hspace*{-1cm} f(i,j) = \max\left( 
      \max_{i \leq k \leq j}\left\{ \sum_{t=i}^k v[t] + \left( \sum_{t=k+1}^{j} v[t] - f(k+1,j) \right) \right\},  
      \max_{i \leq k \leq j}\left\{ \sum_{t=k}^j v[t] + \left( \sum_{t=i}^{k-1} v[t] - f(i,k-1) \right) \right\} 
    \right) $$

Con $v$ el vector con los valores de las cartas. Si desarrollamos esta funci\'on nos queda la siguiente expresi\'on. 

\begin{eqnarray*}\hspace*{-4cm}
  f(i,j) &=& \max\left( 
      \max_{i \leq k \leq j}\left\{ \sum_{t=i}^k v[t] + \left( \sum_{t=k+1}^{j} v[t] - f(k+1,j) \right) \right\},  
      \max_{i \leq k \leq j}\left\{ \sum_{t=k}^j v[t] + \left( \sum_{t=i}^{k-1} v[t] - f(i,k-1) \right) \right\} 
    \right) \\
    &=& \max \left(
      \max_{i \leq k \leq j}\left\{ \sum_{t=i}^j v[t] - f(k+1,j) \right\},  
      \max_{i \leq k \leq j}\left\{ \sum_{t=i}^j v[t] - f(i,k-1) \right\} 
    \right) \\
    &=& \sum_{t=i}^j v[t] + \max \left(
      \max_{i \leq k \leq j}\left\{ - f(k+1,j) \right\},  
      \max_{i \leq k \leq j}\left\{ - f(i,k-1) \right\} 
    \right) \\
    &=& \sum_{t=i}^j v[t] - \min \left(
      \min_{i \leq k \leq j}\left\{ f(k+1,j) \right\},  
      \min_{i \leq k \leq j}\left\{ f(i,k-1) \right\} 
    \right) \\
    &=& \sum_{t=i}^j v[t] - \min_{i \leq k \leq j}\left\{ \min \left( f(k+1,j), f(i,k-1) \right) \right\} 
\end{eqnarray*}

Veamos que el caso base es cuando hay dos cartas solamente, y $f(i,j)$ queda como 

\begin{eqnarray*}
f(1,2) &=& v[1]+v[2] - \min( \min( f(2,2), f(1,0) ), \min( f(3,2), f(1,1) ) ) \\
       &=& v[1]+v[2] - \min( f(2,2), f(1,1) ) \\
       &=& v[1]+v[2] - \min( v[2], v[1] ) = \max( v[1], v[2] )
\end{eqnarray*}

Dado el caso base de las dos cartas, y la funci\'on recursiva $f(i,j)$, la funci\'on que devuelve la soluci\'on \'optima dadas las $n$ cartas es $f(1,n)$. 

\subsection{An\'alisis de complejidad}

\subsection{Test de complejidad}

\subsection{Compilar y ejecutar}
