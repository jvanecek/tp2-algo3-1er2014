\section{Problema 1: Roban\'umeros}

\subsection{Presentaci\'on del problema}

\subsection{Resoluci\'on}

\subsubsection{1era solucion}
En cada mano, agarrar la cantidad de cartas que sumen mas. 

No funciona. Contraejemplo: $[-1, -6, -10]$
  
\subsubsection{2da solucion}

Sea $v$ un vector de $n$ enteros que son los valores de las cartas, y dada la funcion recursiva: 

\begin{equation*}\hspace*{-2cm}
 f(i,j,p) = \max \left( 
  \underbrace{ \max_{1\leq k \leq j}{ \left\{ \left( \sum_{t=i}^k v_t \right) p + f\left(k,j, (p+1)\bmod{2} \right) \right\} } }_{ \text{Lo mejor de la izquierda} },
  \underbrace{ \max_{1\leq k \leq j}{ \left\{ \left( \sum_{t=k}^j v_t \right) p + f\left(i,k, (p+1)\bmod{2} \right) \right\} } }_{ \text{Lo mejor de la derecha} }
 \right) 
\end{equation*}

La soluci\'on al problema es $f(1,n,0)$ donde el 3er par\'ametro puede ser 0 o 1, y representa al jugador

% \left{ \right} 

\subsubsection{3ra solucion}

La solucion es el valor $\max (f_{\rightarrow}(1,n,yo), f_{\leftarrow}(1,n,yo))$ donde las funciones internas se definen como: 

\begin{align*}\hspace*{-2cm}
 f_{\rightarrow}(i,j,p) &= 
 \left\{
    \begin{array}{ll}
	\displaystyle \max_{1\leq k\leq j} \left\{ \sum_{t=i}^k v[t] + \min( f_{\rightarrow}(k+1, j, p.next()), f_{\leftarrow}(k+1,j,p.next() ))  \right\}& \mbox{si } p = yo \\
	\displaystyle \min_{1\leq k\leq j} \left\{ \max( f_{\rightarrow}(k+1, j, p.next()), f_{\leftarrow}(k+1,j,p.next() ) \right\} & \mbox{si } p = tu
    \end{array}
\right. \\
 f_{\leftarrow}(i,j,p) &= 
 \left\{
    \begin{array}{ll}
	\displaystyle \max_{1\leq k\leq j} \left\{ \sum_{t=k}^j v[t] + \min( f_{\rightarrow}(i, k-1, p.next()), f_{\leftarrow}(i, k-1,p.next() ))  \right\}& \mbox{si } p = yo \\
	\displaystyle \min_{1\leq k\leq j} \left\{ \max( f_{\rightarrow}(i, k-1, p.next()), f_{\leftarrow}(i, k-1,p.next() ) \right\} & \mbox{si } p = tu
    \end{array}
\right. \\
\end{align*}

\newpage
\subsubsection{4ta solucion}

Defino la funcion $f(i,j)$ como la solucion optima usando de las cartas $i$ a $j$. 

Esto es lo maximo que puedo agarrar con las cartas de la izquierda o de la derecha. 

Supongamos que por la izquierda lo mejor que puedo hacer es usando las primeras $k$ cartas. Significa que el valor que puedo tomar es $\sum_{t=i}^{k} carta[t]$ m\'as lo que me deja tomar el otro jugador (que va a jugar optimamente) en la mitad $[k+1, j]$. Esto es el total que suma las cartas en la mitad ya dicha, menos $f(k+1, j)$, ya que es el valor optimo. La funci\'on queda asi: 

\begin{eqnarray*}\hspace*{-4cm}
  f(i,j) &=& \max\left( 
      \max_{i \leq k \leq j}\left\{ \sum_{t=i}^k v[t] + \left( \sum_{t=k+1}^{j} v[t] - f(k+1,j) \right) \right\},  
      \max_{i \leq k \leq j}\left\{ \sum_{t=k}^j v[t] + \left( \sum_{t=i}^{k-1} v[t] - f(i,k-1) \right) \right\} 
    \right) \\
    &=& \max \left(
      \max_{i \leq k \leq j}\left\{ \sum_{t=i}^j v[t] - f(k+1,j) \right\},  
      \max_{i \leq k \leq j}\left\{ \sum_{t=i}^j v[t] - f(i,k-1) \right\} 
    \right) \\
    &=& \sum_{t=i}^j v[t] + \max \left(
      \max_{i \leq k \leq j}\left\{ - f(k+1,j) \right\},  
      \max_{i \leq k \leq j}\left\{ - f(i,k-1) \right\} 
    \right) \\
    &=& \sum_{t=i}^j v[t] - \min \left(
      \min_{i \leq k \leq j}\left\{ f(k+1,j) \right\},  
      \min_{i \leq k \leq j}\left\{ f(i,k-1) \right\} 
    \right) \\
    &=& \sum_{t=i}^j v[t] - \min_{i \leq k \leq j}\left\{ \min \left( f(k+1,j), f(i,k-1) \right) \right\} 
\end{eqnarray*}


\subsection{Demostraci\'on}

\subsection{An\'alisis de complejidad}

\subsection{Test de complejidad}

\subsection{Compilar y ejecutar}
