\section{Problema 1: Roban\'umeros}

\subsection{Presentaci\'on del problema}

\subsection{Resoluci\'on}

\subsubsection{1era solucion}
En cada mano, agarrar la cantidad de cartas que sumen mas. 

No funciona. Contraejemplo: $[-1, -6, -10]$
  
\subsubsection{2da solucion}

Sea $v$ un vector de $n$ enteros que son los valores de las cartas, y dada la funcion recursiva: 

\begin{equation*}
 \hspace*{-2cm}
 f(i,j,p) = \max \left( 
  \underbrace{ \max_{1\leq k \leq j}{ \left\{ \left( \sum_{t=i}^k v_t \right) p + f\left(k,j, (p+1)\bmod{2} \right) \right\} } }_{ \text{Lo mejor de la izquierda} },
  \underbrace{ \max_{1\leq k \leq j}{ \left\{ \left( \sum_{t=k}^j v_t \right) p + f\left(i,k, (p+1)\bmod{2} \right) \right\} } }_{ \text{Lo mejor de la derecha} }
 \right) 
\end{equation*}

La soluci\'on al problema es $f(1,n,0)$ donde el 3er par\'ametro puede ser 0 o 1, y representa al jugador

% \left{ \right} 

\subsection{Demostraci\'on}

\subsection{An\'alisis de complejidad}

\subsection{Test de complejidad}

\subsection{Compilar y ejecutar}
