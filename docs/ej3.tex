\section{Problema 3: Saltos en {\it La Matrix}}

\subsection{Presentaci\'on del problema}

La Matrix es un juego que consite de participantes en un tablero cuadrado de $n\times n$. Cada casillero contiene un resorte que le permite al jugador saltar a otro casillero en direcci\'on vertical u horizontal dependiendo del valor de salto del resorte. Adem\'as cada participante cuenta con una cantidad $k$ de potencias extra que puede usar para llegar a casilleros mas alejados, y que se van agotando a medida que las vamos usando.

El problema esta en encontrar la cantidad m\'inima de saltos que puede hacer un participante desde un casillero origen a otro destino, devolver la secuencia de casilleros por los que pasa, y hacerlo en un tiempo de $O(n^3 k)$. 

\subsection{Resoluci\'on}

El problema lo modelamos con un grafo, donde cada casillero es un nodo del mismo, y existe arista entre dos nodos, si y solo si son alcanzables en un salto. Ahora bien, para determinar si existe un salto entre dos casilleros hay que tener en cuenta las potencias extras que poseemos al momento de realizar el salto. A pesar de ello, las aristas no tienen pesos, y por eso usando un algoritmo como BFS podemos determinar la distancia m\'inima entre dos nodos y es este el que vamos a usar para encontrar la soluci\'on al problema propuesto. 

Asique lo que usamos es un BFS sobre un grafo \textit{din\'amico}, ya que el conjunto de nodos adyacentes a un nodo $v$ va a ir cambiando dependiendo de las potencias que tenga en ese momento, aunque como mucho puede en un casillero puedo saltar a todos los de arriba, a todos los de abajo y a todos los de los costados. Es decir, a grosso modo, $d(v) \leq 2(n-1)$. 

Veamos que la complejidad pedida tambi\'en se cumple, ya que el grafo tiene $n^2$ nodos, y las aristas son: 

\begin{equation*}
 |E| = \sum_{v \in V} d(v) / 2 \leq \sum_{v \in V} 2(n-1) / 2 = n^2 (n-1) = n^3 - n^2
\end{equation*}

Por lo tanto, la complejidad del BFS queda en $O(|V| + |E|) \leq O(n^2 + n^3 - n^2) = O(n^3) \in O(n^3 k)$. 

\subsection{Implementaci\'on}

El algoritmo lo escribimos en Java, y contamos con dos clases: \code{Nodo} y \code{Tablero}. La primera cuenta con las siguientes variables: 

\begin{itemize}
 \item \code{int fila} y \code{col}: fila y columna del casillero.
 \item \code{int poderes\_disponibles}: variable que consulto cuando pido tomo sus adyacentes durante el BFS
 \item \code{int poderes\_usados}: variable que consulto al momento de construir la soluci\'on final. 
 \item \code{Nodo parent}: nodo padre en el \'arbol que arma BFS. Es null si el nodo en cuesti\'on es el root, o lo que es lo mismo, el nodo origen. 
\end{itemize}

Por otro lado, \code{Tablero} cuenta con las siguientes variables: 

\begin{itemize}
 \item \code{int n}: Cantidad de filas (y columnas)
 \item \code{int[][] potencias}: cantidad fija de potencias para cada casillero
 \item \code{boolean[][] recorridos}: a medida que BFS recorre el arbol de nodos, va seteando esta matriz sobre los nodos recorridos. 
 \item \code{Nodo src}: nodo origen, o casillero inicial. 
 \item \code{Nodo dst}: nodo destino, o casillero final. 
 \item \code{List<Nodo> solucion}: lista de nodos (o casilleros) en orden pertenecientes al camino m\'inimo entre \code{src} y \code{dst}
\end{itemize}

El BFS implementado funciona de la misma manera que \'el com\'un: arranco con el nodo inicial, recorro todos los adyacentes buscando el nodo final, si no lo encuentra, busca en todos los adyacentes que recorri\'o en la iteraci\'on anterior, y as\'i sucesivamente. Como vimos en la pr\'actica, BFS recorre todos los nodos del \'arbol y por lo tanto, si \code{dst} se encuentra en el mismo, nuestro algoritmo termina. 

Ahora queremos ver que no nos salteamos ninguna soluci\'on, y para ello veamos que siempre recorremos todas las posibilidades cuando buscamos los nodos adyacentes. Supongamos que un momento dado queremos encontrar los vecinos de un nodo $v_i$, la potencia del mismo que es $p_i$ y la potencia extra disponible (\code{poderes\_disponibles}) es $k_i$. Los casilleros a los que puedo llegar usando solamente $p_i$, van a seguir teniendo como \code{poderes\_disponibles} a $k_i$, en cambio, a los que llegu\'e usando $q$ poderes extra, van a tener como \code{poderes\_disponibles} a $k_i-q$. Adem\'as todos estos nodos adyacentes van a tener como \code{parent} a $v_i$. 

Nuestro BFS empieza con $v_1$ (\code{src}), cuya potencia fija es $p_1$, y su variable \code{poderes\_disponibles} seteado en $k$. Asique, al final, estamos cubriendo todos los posibles nodos adyacentes, y al final BFS recorre todos los posibles caminos que existen entre dos nodos cualesquiera. 

Una vez que encontramos a \code{dst} (sabemos que lo hace porque BFS termina), nuestro algoritmo ``sube'' en el \'arbol que construyo BFS a trav\'es de la variable de los nodos \code{parent} que apunta a su nodo padre, y va calculando la potencia extra usada en cada salto, ya que si un nodo $v_{i+1}$ tiene una \code{potencia\_disponible} de $k_{i+1}$ y su padre tiene una potencia de $k_i$, entonces la potencia extra (\code{poderes\_usados}) que us\'o el padre fue $k_{i+1} - k_i$.

\subsection{Algoritmo}

A continuaci\'on escribiremos en pseudoc\'odigo el BFS y la construcci\'on de la soluci\'on

\begin{algorithm}[H]
\caption{int resolver()}
\begin{algorithmic}[1]
  \STATE return bfs([src], 1);
\end{algorithmic}
\end{algorithm}

\begin{algorithm}[H]
\caption{\texttt{bfs}(List$<$Nodo$>$, int dist\_root)}
\begin{algorithmic}[1]
  \STATE List$<$Nodo$>$ nodos\_siguientes = [];
  \FOR{\textbf{each} Nodo v \textbf{in} nodos}
    \IF{ \textbf{not} fueRecorrido(v) }
      \IF{ v == dst }
        \STATE construirSolucion(v);
        \RETURN dist\_root;
      \ELSE
        \STATE marcarComoRecorrido(v);
        \STATE nodos\_siguientes.addAll( getAdyacentes(v) );
      \ENDIF
    \ENDIF
  \ENDFOR
  \RETURN bfs(nodos\_siguientes, dist\_root++);
\end{algorithmic}
\end{algorithm}

\subsection{Test de complejidad}

Como ya calculamos anteriormente en la secci\'on \textbf{Resoluci\'on}, la complejidad de nuestro algoritmo es $O(n^3)$, acotando la cantidad de adyacentes que puede tener un nodo por $2(n-1)$. 

Aunque si queremos tener en cuenta a $k$, podemos considerar constante los $p$ y estimar que $d(v)$ va a ser menor o igual a $4k$, para cualquier $v$, ya que como mucho va a haber una arista a cada $k$ nodo en las cuatro direcciones. Luego, las cantidad de aristas del grafo del tablero es

\begin{equation*}
 |E| = \sum_{v \in V} d(v) / 2 \leq \sum_{v \in V} 4k / 2 = n^2 2k
\end{equation*}

Y la complejidad de BFS teniendo en cuenta a $k$ queda en $O(|V|+|E|) \leq O(n^2 + n^2 k) \leq O(n^2k) \in O(n^3k)$. 

Para evaluar la complejidad analizamos tres casos: 

\begin{enumerate}
 \item Primero estudiamos el impacto del tama\~no del tablero en nuestro algoritmo, teniendo un $p$ constante m\'inimo, y sin potencia adicional. 
 
 \item Segundo, siendo el tama\~no constante, como influye la cantidad de potencia adicional. 
 
 \item Por \'ultimo, contrastamos el impacto de $k$ en tablero de distintos tama\~nos. 
\end{enumerate}

\ponerGrafico{./ej3/caso_uno.jpg}{Tiempo usando un tablero con $p=1$ y $k=0$}{0.6}{}
Como era esperable, en funci\'on del tama\~no, el algoritmo resuelve el problema en un orden cuadrado. 

\ponerGrafico{./ej3/caso_tres_lineal.jpg}{En funci\'on de $k$, fijando $n=100$}{0.6}{}

Este caso tambi\'en confirmamos nuestros hip\'otesis de que en funci\'on del $k$, la complejidad es lineal. 

\ponerGrafico{./ej3/caso_dos_plot.jpg}{Contraste de distintos tama\~nos, en funci\'on de $k$}{0.6}{}

Como \'ultimo experimento quisimos constrastamo el impacto del $k$ en distintos tama\~nos, y si bien todos un crecimiento proporcional, a mayor tama\~no m\'as tiempo le toma al algoritmo encontrar la soluci\'on, y esto tiene sentido porque hay mayor cantidad de vecinos que recorrer. 

\subsection{C\'odigo fuente}

\subsubsection{\code{Nodo}}

Constructores de la clase: 

\footnotesize\footnotesize\begin{verbatim}
public Nodo(int i, int j, int k){
  this.parent = null;
  this.fila = i; 
  this.col = j; 
  this.poderes_disponibles = k;
}

public Nodo(int i, int j, int k, Nodo v){
  this.parent = v;
  this.fila = i; 
  this.col = j; 
  this.poderes_disponibles = k;
}
\end{verbatim}\normalsize\normalsize

\subsubsection{\code{Tablero}}

M\'etodo de inicializaci\'on del Tablero: 

\footnotesize\begin{verbatim}
public Tablero(int n, int k, int f0, int c0, int fn, int cn, int[][] potencias){
  this.n = n;
  this.src = Nodo(f0, c0, k);
  this.dst = Nodo(fn, cn, k);
  this.potencias = potencias;
  this.recorridos = new boolean[n][n];
  this.solucion = [];
}
\end{verbatim}\normalsize

La complejidad es copiar la matriz de potencias, que tiene una complejidad de $O(n^2)$ operaci\'ones. 

\footnotesize\begin{verbatim}
public boolean estaEnTablero( int fila, int col ){
  return fila >= 1 && col >= 1 &&
         fila <= n && col <= n;
}
\end{verbatim}\normalsize

Complejidad: $O(1)$


\footnotesize\begin{verbatim}
public List<Nodo> getAdyacentes(Nodo v){
  List<Nodo> ady = [];

  int p = getPotencia(v);
  int k = v.getPoderesDisponibles();
  int f = v.getFila();
  int c = v.getCol();

  // agrego todos los vecinos a los que llego con la potencia fija
  for( int i = 1 ... p ){
    if( estaEnTablero(f-i,c) ) ady.add( new Nodo(f-i,c,k,v) ); // abajo
    if( estaEnTablero(f+i,c) ) ady.add( new Nodo(f+i,c,k,v) ); // arriba
    if( estaEnTablero(f,c-i) ) ady.add( new Nodo(f,c-i,k,v) ); // a la izq
    if( estaEnTablero(f,c+i) ) ady.add( new Nodo(f,c+i,k,v) ); // a la der
  }

  // agrego todos los vecinos a los que llego con la potencia extra
  for( int i = 1 ... k ){
    if( estaEnTablero(f-(p+i),c) ) ady.add( new Nodo(f-(p+i),c,k-i,v) ); // abajo;
    if( estaEnTablero(f+(p+i),c) ) ady.add( new Nodo(f+(p+i),c,k-i,v) ); // arriba
    if( estaEnTablero(f,c-(p+i)) ) ady.add( new Nodo(f,c-(p+i),k-i,v) ); // a la izq
    if( estaEnTablero(f,c+(p+i)) ) ady.add( new Nodo(f,c+(p+i),k-i,v) ); // a la der
  }
  return ady;
}
\end{verbatim}\normalsize

Complejidad: $O(p+k) \in O(n)$

\footnotesize\begin{verbatim}
public int bfs(List<Nodo> nodos, int dist_root){
  List<Nodo> nodos_siguientes = new Vector<Nodo>();

  foreach( Nodo v in nodos ){
    if( !fueRecorrido(v) ){
      if( v == dst ){
        construirSolucion(v);
        return dist_root;
      }else{
        marcarComoRecorrido(v);
        nodos_siguientes.addAll( getAdyacentes(v) );
      }
    }
  }

  return bfs(nodos_siguientes, dist_root++);
}
\end{verbatim}\normalsize

Recorre todos los nodos del grafo una sola vez, y por cada uno pide los adyacentes, cuya operaci\'on cuesta $O(n)$. En total, la complejidad de este m\'etodo es $n^2 \times O(n) = O(n^3)$.  

\footnotesize\begin{verbatim}
public int resolver(){
  List<Nodo> nodos = [];
  nodos.add(src);
  return bfs(nodos, 1);
}
\end{verbatim}\normalsize

Complejidad: $O(n^3)$

\footnotesize\begin{verbatim}
public void construirSolucion(Nodo dst){
  dst.setPoderesUsados(0);
  int poderesDisponiblesSig = dst.getPoderesDisponibles();
  solucion.add(dst);

  while( dst.getParent() != null ){
    dst = dst.getParent();
    dst.setPoderesUsados( dst.getPoderesDisponibles() - poderesDisponiblesSig);
    poderesDisponiblesSig = dst.getPoderesDisponibles();
    solucion.add(0, dst); // agrego al principio. 
  }
}
\end{verbatim}\normalsize

Complejidad: $O(n)$

\subsection{Adicionales}

\subsubsection{Modificaciones del Algoritmo}

La modificaci\'on del juego \textit{La Matrix} consiste en agregarle un costo proporcional al salto a cada resorte. Los jugadores aun tienen $k$ potencias que pueden utilizar para agregar gratuitamente potencias a su salto.

Se nos pide modificar el algoritmo para llegar de un origen a destino con la m\'inima cantidad de costo posible.

El algoritmo anterior se resuelve usando BFS sobre un grafo din\'amico ya que las aristas no tienen costo, en este problema, vamos a usar Dijkstra para remplazar al BFS en la b\'usqueda del camino menos costoso.

La idea en Dijkstra es ir explorando todos los caminos m\'as cortos que parten del v\'ertice origen y que llevan a todos los dem\'as v\'ertices; cuando se obtiene el camino m\'as corto desde el v\'ertice origen, al resto de v\'ertices que componen el grafo, el algoritmo se detiene.

En este juego, surge el problema de que puedo tener m\'as de un costo para llegar de un nodo a otro dependiendo de las unidades de $k$ que utilice, es por eso que proponemos \textit{clonar} el grafo de $n^2$ nodos $k$ veces, formando un grafo tridimensional, para tener control de las potencias utilizadas que pertenecen a $k$ para llegar a un nodo. Con esta estructura, puedo asignar los vecinos de un nodo sabiendo cuantas potencias gratuitas quedan y el camino m\'inimo del origen a destino es el m\'inimo de los caminos a todos los \textit{clones} de destino.

\subsubsection{Modificaciones de la Complejidad}

La complejidad depende principalmente del Algoritmo de Dijkstra, que se ejecuta sobre un grafo de $n^2k$ nodos. Si lo implementamos utilizando una cola de prioridad, Dijkstra tiene complejidad de $O(|V|\log(|V|))$ donde $V$ son los vertices del grafo, entonces la complejidad queda en $O(n^2k \log(n^2k))$.

El nuevo algoritmo tambi\'en tiene la b\'usqueda de un m\'inimo en un arreglo de $k$ elementos para devolver la soluci\'on final, esta tiene complejidad $O(\log k)$ si lo hacemos con busqueda binaria, aunque no afecta a la complejidad final ya que sumada a la de Dijkstra, esta \'ultima prevalece.




